\begin{table}[htbp]\centering
 \footnotesize 
% \caption{Regression Discontinuity Estimates}
\begin{tabular}{lccccc}
\hline\hline
Method & Coef. & Std. Err. & z & P$>|$z$|$ & [95\% C.I.] \\ 
\hline \hline  
Conventional & -0.358 & 0.114 & -3.130 & 0.002 & [-0.582,-0.134] \\ 
 Bias-Corrected & -0.335 & 0.114 & -2.929 & 0.003 & [-0.559,-0.111] \\ 
Robust & -0.335 & 0.191 & -1.755 & 0.079 & [-0.710,0.039] \\ 
  \hline\hline
\end{tabular}
\label{table:rd}
\begin{tablenotes} 
  \justifying \tiny \textbf{Note: }    
   The analysis is based on a sample of 99 observations, with 54 located within the area of exploration announcement ($A^{T}$) and 45 located near the border but outside the area of exploration  ($A^{c}$). 
           We employ a Triangular kernel and the variance-covariance matrix estimator is computed with nearest neighbor variance estimator for heteroskedasticity-robust. The global polynomial fit in  $A^{T}$ and $A^{c}$ is of order 1, the bandwith where the global polynomial fit is of 1399.224 meters.. We estimate all coefficients using conventional, bias-corrected, and robust estimators, and we cluster standard errors at the school level. \end{tablenotes} 
 \end{table} 
