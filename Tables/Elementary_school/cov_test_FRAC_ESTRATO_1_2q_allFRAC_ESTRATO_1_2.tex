\begin{table}[htbp]\centering
 \footnotesize 
% \caption{Regression Discontinuity Estimates}
\begin{tabular}{lccccc}
\hline\hline
Method & Coef. & Std. Err. & z & P$>|$z$|$ & [95\% C.I.] \\ 
\hline \hline  
Conventional & -0.058 & 0.016 & -3.606 & 0.000 & [-0.090,-0.027] \\ 
 Bias-Corrected & -0.070 & 0.016 & -4.345 & 0.000 & [-0.102,-0.038] \\ 
Robust & -0.070 & 0.021 & -3.361 & 0.001 & [-0.111,-0.029] \\ 
  \hline\hline
\end{tabular}
\label{table:rd}
\begin{tablenotes} 
  \justifying \tiny \textbf{Note: }    
   The analysis is based on a sample of 5158 observations, with 3157 located within the area of exploration announcement ($A^{T}$) and 2001 located near the border but outside the area of exploration  ($A^{c}$). 
           We employ a Triangular kernel and the variance-covariance matrix estimator is computed with nearest neighbor variance estimator for heteroskedasticity-robust. The global polynomial fit in  $A^{T}$ and $A^{c}$ is of order 2, the bandwith where the global polynomial fit is of 4395.556 meters.. We estimate all coefficients using conventional, bias-corrected, and robust estimators, and we cluster standard errors at the school level. \end{tablenotes} 
 \end{table} 
