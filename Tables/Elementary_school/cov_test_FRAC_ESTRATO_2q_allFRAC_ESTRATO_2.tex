\begin{table}[htbp]\centering
 \footnotesize 
% \caption{Regression Discontinuity Estimates}
\begin{tabular}{lccccc}
\hline\hline
Method & Coef. & Std. Err. & z & P$>|$z$|$ & [95\% C.I.] \\ 
\hline \hline  
%   Conventional & %   -0.007 & %   0.013 & %   -0.525 & %   0.600 & %   [-0.032,0.018] \\ 
 %   Bias-Corrected & %   -0.015 & %   0.013 & %   -1.176 & %   0.240 & %   [-0.040,0.010] \\ 
Robust & -0.015 & 0.019 & -0.799 & 0.424 & [-0.052,0.022] \\ 
  \hline\hline
\end{tabular}
\label{table:rd}
\begin{tablenotes} 
  \justifying \tiny \textbf{Note: }    
   The analysis is based on a sample of 5381 observations, with 3335 located within the area of exploration announcement ($A^{T}$) and 2046 located near the border but outside the area of exploration  ($A^{c}$). 
           We employ a Triangular kernel and the variance-covariance matrix estimator is computed with nearest neighbor variance estimator for heteroskedasticity-robust. The global polynomial fit in  $A^{T}$ and $A^{c}$ is of order 1, the bandwith where the global polynomial fit is of 4461.649 meters.. We estimate robust estimators, and we cluster standard errors at the school level. \end{tablenotes} 
 \end{table} 
