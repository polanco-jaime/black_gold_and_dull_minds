\begin{table}[htbp]\centering
 \footnotesize 
% \caption{Regression Discontinuity Estimates}
\begin{tabular}{lccccc}
\hline\hline
Method & Coef. & Std. Err. & z & P$>|$z$|$ & [95\% C.I.] \\ 
\hline \hline  
Conventional & -0.052 & 0.032 & -1.616 & 0.106 & [-0.115,0.011] \\ 
 Bias-Corrected & -0.032 & 0.032 & -0.999 & 0.318 & [-0.095,0.031] \\ 
Robust & -0.032 & 0.045 & -0.708 & 0.479 & [-0.121,0.057] \\ 
  \hline\hline
\end{tabular}
\label{table:rd}
\begin{tablenotes} 
  \justifying \tiny \textbf{Note: }    
   The analysis is based on a sample of 1008 observations, with 501 located within the area of exploration announcement ($A^{T}$) and 507 located near the border but outside the area of exploration  ($A^{c}$). 
           We employ a Triangular kernel and the variance-covariance matrix estimator is computed with nearest neighbor variance estimator for heteroskedasticity-robust. The global polynomial fit in  $A^{T}$ and $A^{c}$ is of order 1, the bandwith where the global polynomial fit is of 3904.328 meters.. We estimate all coefficients using conventional, bias-corrected, and robust estimators, and we cluster standard errors at the school level. \end{tablenotes} 
 \end{table} 
