\begin{table}[htbp]\centering
 \footnotesize 
% \caption{Regression Discontinuity Estimates}
\begin{tabular}{lccccc}
\hline\hline
Method & Coef. & Std. Err. & z & P$>|$z$|$ & [95\% C.I.] \\ 
\hline \hline  
Conventional & -0.022 & 0.065 & -0.337 & 0.736 & [-0.149,0.105] \\ 
 Bias-Corrected & -0.038 & 0.065 & -0.594 & 0.553 & [-0.165,0.088] \\ 
Robust & -0.038 & 0.094 & -0.408 & 0.683 & [-0.223,0.146] \\ 
  \hline\hline
\end{tabular}
\label{table:rd}
\begin{tablenotes} 
  \justifying \tiny \textbf{Note: }    
   The analysis is based on a sample of 230 observations, with 130 located within the area of exploration announcement ($A^{T}$) and 100 located near the border but outside the area of exploration  ($A^{c}$). 
           We employ a Triangular kernel and the variance-covariance matrix estimator is computed with nearest neighbor variance estimator for heteroskedasticity-robust. The global polynomial fit in  $A^{T}$ and $A^{c}$ is of order 1, the bandwith where the global polynomial fit is of 3116.911 meters.. We estimate all coefficients using conventional, bias-corrected, and robust estimators, and we cluster standard errors at the school level. \end{tablenotes} 
 \end{table} 
